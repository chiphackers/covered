\section{Section 7.  Test and Checkout Procedure}\label{page_testing}
 \begin{Desc}
\item[Section 7.1.  Testing Methodology]\par
 Testing the Covered tool for general \char`\"{}goodness\char`\"{}, which is required for release, is  accomplished with its own suite of C and Verilog diagnostics. These suite of tests are run  in a regression manner; that is, each diagnostic is self-checking and run in serial order.  The results of each diagnostic are output to standard output as well as an output file. After all diagnostics are run, the output file is grep'ed for the {\bf keyword} {\rm (p.\,\pageref{structkeyword})} \char`\"{}PASSED\char`\"{}. The number of diagnostics finishing the PASS message are compared against the total number of diagnostics. The results of which are output to standard output.

 To release a new version of the tool for general consumption, the following testing procedures are required to occur prior to the release.\begin{enumerate}
\item 
New C/Verilog diagnostics are written to test new features of tool. These diagnostics will be self-contained and self-checking, displaying a message of \char`\"{}PASSED\char`\"{} if the diagnostic has successfully tested the feature under test or some message displaying the cause of failure. The failure message may not contain the {\bf keyword} {\rm (p.\,\pageref{structkeyword})} \char`\"{}PASSED\char`\"{} in its description.\item 
These newly written diagnostics are added to the regression suite, the list of which is maintained in the Makefile located in the diagnostic directory.\item 
A regression run is run in both the C diagnostic directory as well as the Verilog diagnostic directory.\item 
100\% of the diagnostics in the regression suite result in a PASS message for both the C and Verilog directories.\end{enumerate}
\end{Desc}


\begin{Desc}
\item[Section 7.2.  Testing Directories]\par
 The reason for having two directories for regression testing relies on the feature under test. Verilog diagnostics are condensed DUTs which only contain the required code for testing a particular syntax of the Verilog language to verify that Covered is able to correctly parse the code and generate the appropriate coverage results for that feature. All Verilog diagnostics are accompanied by a text file that is used for comparison purposes. The Makefile, after simulating the Verilog file, creating the dumpfile, generating the CDD and generating a verbose report based on the CDD will compare the generated report to the text file by performing a UNIX \char`\"{}diff\char`\"{} command. If the results of the \char`\"{}diff\char`\"{} are no differences between the two files, the Makefile will assume that the diagnostic has successfully passed and output the {\bf keyword} {\rm (p.\,\pageref{structkeyword})} \char`\"{}PASSED\char`\"{} to the output result file. If the results of \char`\"{}diff\char`\"{} show that there are differences between the two files, the Makefile will assume failure and output the {\bf keyword} {\rm (p.\,\pageref{structkeyword})} \char`\"{}FAILED\char`\"{} to the output result file.

 C diagnostics exist to test certain functions of Covered rather than the entire tool itself. Many times it is impossible/impractical to create Verilog diagnostics that exercise certain functions within Covered to completion. In these cases, it is often easier to write more specialized tests that can more quickly manipulate inputs to functions and verify that all output values are correct. Examples of C diagnostics that currently exist in the C  regression directory include tests of bitwise operators, mathematical functions, etc.

 It is suggested that if functions can be adequately tested at a system level, that it be done so using Verilog diagnostics as these will get the most testing out of the entire tool. However, if functions are best tested in seclusion, it is suggested that the C testing environment be used.\end{Desc}


\begin{Desc}
\item[Section 7.3.  Verilog Testing Procedure]\par
 {\bf TBD}\end{Desc}


\begin{Desc}
\item[Go To Section...]\par
\begin{CompactItemize}
\item 
{\bf Section 1.  Introduction} {\rm (p.\,\pageref{page_intro})}\item 
{\bf Section 2.  Project Plan} {\rm (p.\,\pageref{page_project_plan})}\item 
{\bf Section 3.  Coding Style Guidelines} {\rm (p.\,\pageref{page_code_style})}\item 
{\bf Section 4.  Development Tools} {\rm (p.\,\pageref{page_tools})}\item 
{\bf Section 5.  Project \char`\"{}Big Picture\char`\"{}} {\rm (p.\,\pageref{page_big_picture})}\item 
{\bf Section 6.  Coverage Development Reference} {\rm (p.\,\pageref{page_code_details})}\item 
{\bf Section 8.  Odds and Ends Information} {\rm (p.\,\pageref{page_misc})}\end{CompactItemize}
\end{Desc}
